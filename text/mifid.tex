\def \boolans {\\ \textbf{Sì \ \ \ No}} 

\appendix 

\section{Questionario MiFid2}

\begin{itemize}

    \item \textbf{Sezione 1: Conoscenza dei servizi di investimento (nel corso degli ultimi 3 anni)}

    \begin{enumerate}

        \item Prima parte

        \begin{enumerate}[label*=\arabic*.]
            \item Servizi collegati alla consulenza in materia d'investimento.
            \boolans
            \item Conti correnti con dossier titoli.
            \boolans
            \item Investimenti in strumenti del risparmio gestito (Fondi, ETF, Gestioni ecc.).
            \boolans
            \item Trasmissioni di ordini di acquisto e vendita di strumenti finanziari su deposito amministrato.
            \boolans
            \item Servizi informatici offerti dagli intermediari per i propri investimenti (online banking, trading online).
            \boolans
        \end{enumerate}

        \item Seconda parte

        \begin{enumerate}[label*=\arabic*.]
            \item Conoscenza di strumenti finanziari generica.
            \boolans
            \item Strumenti monetari (es. Buoni Ordinari del Tesoro, Fondi monetari).
            \boolans
            \item Strumenti obbligazionari di Stato e sopranazionali (es. BTP, Fondi obbligazionali).
            \boolans
            \item Strumenti obbligazionari corporate/societari (es. Obbligazioni bancarie, Obbligazioni Enel, Eni, Fondi obbligazionari corporate).
            \boolans
            \item Strumenti azionari (azioni quotate in Borsa, Fondi bilanciati o azionari).
            \boolans
            \item Strumenti finanziari complessi (obbligazionari con componente strutturata, warrant, convertibili).
            \boolans
            \item Contratti Futures, opzioni, swap e altri strumenti derivati.
            \boolans
        \end{enumerate}

    \end{enumerate}

    \item \textbf{Sezione 2: Esperienza nei servizi di investimento (nel corso degli ultimi 3 anni)}

    \begin{enumerate}[label*=\arabic*.]
        \item Prima parte

        \begin{enumerate}[label*=\arabic*.]
            \item Servizi legati alla consulenza in materia di investimento.
            \boolans
            \item Conti correnti con dossier titoli.
            \boolans
            \item Investimenti in strumenti del risparmio gestito (Fondi, ETF, Gestioni etc).
            \boolans
            \item Trasmissioni di ordini di acquisto e vendita di strumenti finanziari su deposito amministrato.
            \boolans
            \item Servizi informatici offerti dagli intermediari per i propri investimenti (Online banking, trading online).
            \boolans
        \end{enumerate}

        \item Seconda parte

        \begin{enumerate}[label*=\arabic*.]
            \item Ha sottoscritto qualche strumento?
            \boolans
            \item Strumenti monetari (es. Buoni Ordinari del Tesoro, Fondi monetari).
            \boolans
            \item Strumenti obbligazionari di stato e sopranazionali (Es. BTP, Fondi obbligazionari).
            \boolans
            \item Strumenti obbligazionari corporate/societari (Es. Obbligazioni bancarie, Obbligazioni Enel, Eni, Fondi obbligazionari corporate).
            \boolans
            \item Strumenti azionari (Azioni quotate in Borsa, Fondi bilanciati o azionari).
            \boolans
            \item Strumenti finanziari complessi (obbligazionari con componente strutturata, warrant, convertibili).
            \boolans
            \item Contratti Futures, opzioni, swap e altri strumenti derivati
            \boolans
        \end{enumerate}

        \item Terza parte

        \begin{enumerate}[label*=\arabic*.]
            \item Frequenza delle operazioni aventi ad oggetto strumenti finanziari (per operazioni si intende il singolo acquisto e/o vendita di strumenti finanziari).

            \begin{enumerate}
                \item Bassa (fino a 5 operazioni/anno)
                \item Media/bassa (fino a 10 operazioni/anno)
                \item Media (fino a 20 operazioni/anno)
                \item Medio/Alta (fino a 30 operazioni/anno)
                \item Alta (oltre 30 operazioni/anno)
            \end{enumerate}

            \item Strumenti finanziari negoziati/sottoscritti (volumi complessivi nel corso dell’ultimo anno, ivi incluso l’investimento indiretto tramite OICR).

            \begin{enumerate}
                \item Bassa (fino a 10000 euro/anno)
                \item Media/bassa (fino a 25000 euro/anno)
                \item Media (fino a 50000 euro/anno)
                \item Medio/Alta (fino a 200000 euro/anno)
                \item Alta (oltre 200000 euro/anno)
            \end{enumerate}

        \end{enumerate}

    \end{enumerate}

    \item \textbf{Sezione 3: Situazione economica e finanziaria}

    \begin{enumerate}

        \item Prima parte

        \begin{enumerate}[label*=\arabic*]
            \item Consistenza del reddito lordo (media annua ultimi 3 anni).
            \begin{enumerate}
                \item fino a 40.000
                \item da 40.000 a 80.000
                \item da 80.000 a 10.000
                \item da 100.000 a 250.000
                \item oltre 250.000 
            \end{enumerate}

            \item Propensione e capacità di generare risparmio (in percentuale al reddito annuo).
            \begin{enumerate}
                \item fino al 10\%
                \item fino al 20\%
                \item oltre il 20\%
            \end{enumerate}

            \item Patrimonio mobiliare (consistenza attività liquide, investimenti).
            \begin{enumerate}
                \item fino a 200.000
                \item da 200.000 a 350.000
                \item da 350.000 a 500.000
                \item da 500.000 a 2.000.000
                \item oltre 2.000.000 
            \end{enumerate}

            \item Patrimonio immobiliare (escluso immobile di abitazione).
            \begin{enumerate}
                \item fino a 500.000
                \item da 500.000 a 1.000.000
                \item da 1.000.000 a 2.500.000
                \item da 2.500.000 a 5.000.000
                \item oltre 5.000.000
            \end{enumerate}

            \item Impegni finanziari (passività totali).
            \begin{enumerate}
                \item oltre 1.000.000
                \item da 500.000 a 1.000.000
                \item da 250.000 a 500.000
                \item da 100.000 a 250.000
                \item fino a 100.000
            \end{enumerate}

            \item Su quale percentuale del patrimonio desideri minimizzare le oscillazioni negative o positive derivanti da investimento?
            \begin{enumerate}
                \item fino al 95\%
                \item fino al 70\%
                \item fino al 50\%
                \item fino al 30\%
                \item fino al 20\%
            \end{enumerate}

            \item Professione lavorativa attuale.
            \begin{enumerate}
                \item Non occupato
                \item Pensionato
                \item Lavoratore dipendente
                \item Libero professionista
                \item Imprenditore
            \end{enumerate}
        \end{enumerate}

    \end{enumerate}

    \item \textbf{Sezione 4 - Obiettivi di investimento e profilo di rischio}

    \begin{enumerate}
        
        \item Prima parte

        \begin{enumerate}[label*=\arabic*]
            
            \item Quale tra i seguenti descrive più propriamente il Suo obiettivo di investimento?
            \begin{enumerate}
                \item Conservazione del valore reale del capitale
                \item Investimento della liquidità per futuri impegni finanziari
                \item Rendimento superiore al tasso di inflazione
                \item Elevato rendimento con alto rischio potenziale 
            \end{enumerate}

            \item Entro quale orizzonte temporale colloca gli investimenti del patrimonio sotto consulenza?
            \begin{enumerate}
                 \item breve termine (1 anno) 
                 \item breve-medio termine (1-3 anni) 
                 \item breve-medio termine (3-5 anni)
                 \item medio/lungo termine (5-10 anni) 
                 \item lungo termine (oltre 10 anni)  
            \end{enumerate} 

            \item Ipotizzi di dovere investire il capitale scegliendo tra tre diversi strumenti A, B e C. Come ripartirebbe il suo investimento tra i tre strumenti?
            \todo{Questa domanda ha senso?}
            \begin{enumerate}
                \item Rendimento di A: garantiti 3\%
                \item Rendimento di B: massimo 7\%, minimo 0\%
                \item Rendimento di C: massimo 20\% minimo -10\%
            \end{enumerate}

            \item Quale ritiene essere la Sua capacità oggettiva di sopportare perdite o oscillazioni negative sul patrimonio sotto consulenza nel corso di un anno, tenuto conto delle Sue abitudini e comportamenti rilevati in passato?
            \begin{enumerate}
                \item fino al 5\%
                \item fino al 10\%
                \item fino al 15\%
                \item fino al 30\%
                \item oltre il 30\%
            \end{enumerate}

            \item Ipotizzi di avere investito il capitale in due diversi strumenti A e B; dopo sei mesi il primo investimento A ha fruttato un rendimento del 10\%; il secondo investimento B ha avuto invece una perdita del 10\%. Quali decisioni assume?
            \begin{enumerate}
                \item Vendo lo strumento A e mantengo l’investimento in B in attesa di recuperare le perdite
                \item Incremento l’investimento in B per mediare le perdite e abbassare il costo di carico
                \item Prima di decidere, analizzo le prospettive di rendimento ed i rischi di entrambi gli strumenti 
            \end{enumerate}

            \item Titolo di studio.
            \begin{enumerate}
                \item Nesun titolo di studio conseguito
                \item Licenza media
                \item Diploma
                \item Laurea
                \item Post Laurea
            \end{enumerate}
        \end{enumerate}
    \end{enumerate}

\end{itemize}